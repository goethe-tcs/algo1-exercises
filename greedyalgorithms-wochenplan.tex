% !TeX spellcheck = de_DE
\documentclass{uebung_cs}
\usepackage{algo121}
\blattname{Wochenplan: Gierige Algorithmen}

\begin{document}
\section*{Vorbereitung}
Lies E Kapitel 4 und schau das Video der Woche.

\section*{Dienstag}
\begin{aufgabe}[Dateien auf Band]
    Auf einem Magnetband sind $n$ Dateien gespeichert.
    Die Längen der Dateien sind durch ein Feld $L[1..n]$ gegeben, das heißt, die $i$-te auf dem Band gespeicherte Datei hat Länge $L[i]$.
    Um Datei $k$ zu lesen, muss das Band alle Dateien von $1$ bis $k$ lesen.
    \begin{enumerate}
        \item(\warmup) Betrachte ein Band, auf dem sechs Dateien mit $L=[8,3,1,6,4,2]$ gespeichert sind. Zum Beispiel hat Datei $1$ die Länge $8$.
        Wie hoch sind die Kosten, um Datei $4$ zu lesen?
        \item(\warmup) Wie hoch sind die erwarteten Kosten bei a), wenn ein sechsseitiger Würfel geworfen wird und die damit indizierte Datei gelesen wird?
        \item Wenn jede Datei $k$ mit Wahrscheinlichkeit $1/n$ angefragt wird, sortieren wir die Dateien am Besten der Größe nach. Jetzt aber sind manche Dateien beliebter als andere: Datei $k$ wird mit Wahrscheinlichkeit $F[k]$ angefragt. Das heißt, $F[1..n]$ ist eine Wahrscheinlichkeitsverteilung über $\{1,\dots,n\}$ und die erwarteten Kosten, um Datei~$k$ zu lesen, sind
        \[\sum_{i=1}^k F[i]\cdot L[i]\,.\]
        In welcher Reihenfolge müssen wir die Dateien jetzt sortieren, damit die erwarteten Kosten so klein wie möglich sind?
    \end{enumerate}
\end{aufgabe}


\begin{aufgabe}[Scheduling]
    Gegeben sind Startzeiten $S[1..n]$ und Endzeiten $F[1..n]$ von $n$ Vorlesungen.
    \begin{enumerate}
        \item(\warmup) Beschreibe mathematisch (also durch eine logische Formel, die $S$ und $F$ benutzt), was es heißt, dass die Zeiten von Vorlesung $i$ und $j$ sich überlappen.
        \item Professorin S. Chedule hat folgende Idee, um den gierigen Algorithmus für das Scheduling zu vereinfachen: Anstatt nach den Endzeiten, sortieren wir die Vorlesung nach den Startzeiten, und nehmen also immer die erste Vorlesung in den Stundenplan auf, die als nächstes startet und keinen Konflikt verursacht. Finde ein Gegenbeispiel, in dem dieser Algorithmus fehlschlägt.
    \end{enumerate}
\end{aufgabe}


\begin{aufgabe}[Huffman-Codierung]
    Betrachte das Alphabet $\{\texttt{a},\dots,\texttt{g}\}$ mit der folgenden Häufigkeitstabelle:
	\begin{center}
		\begin{tabular}{ccccccc}
			\texttt{a}&\texttt{b}&\texttt{c}&\texttt{d}&\texttt{e}&\texttt{f}&\texttt{g}\\\hline
			3&3&1&1&7&2&4\\
		\end{tabular}
	\end{center}
    \begin{enumerate}
        \item Führe den Algorithmus zur Huffman-Codierung für diese Häufigkeitstabelle aus und gib jeden Zwischenschritt an.
        \item Gib einen optimalen binären Code an. Gib für jedes Zeichen das zugehörige Codewort an, und male den zugehörigen Binärbaum.
    \end{enumerate}
\end{aufgabe}


\section*{Donnerstag}
\begin{aufgabe}[Stabiles Matching]
\end{aufgabe}

\end{document}
