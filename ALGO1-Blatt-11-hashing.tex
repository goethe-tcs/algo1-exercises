% !TeX spellcheck = de_DE
\documentclass{uebung_cs}
\usepackage{algo123}
\uebung{11}{}{}
\blattname{Wörterbücher, Hashing (Woche 11)}

%%%%%%%%%%%%%%%%%%%%%%%%%%%%%%%%%%%%%%%%%%%%%%%%%%%%%%%%%%%%%%%%%%%%%%%%%%%%

\begin{document}
\textbf{Eigenständige Vorbereitung:}\\
Lies \emoji{book} CLRS Kapitel 11 ohne 11.5 und schau dir das \emoji{television} Video der Woche an.

\textbf{Zeichenlegende:}
\legende{}

%  \schriftlich
%  \bestehen
%  \mittel
%  \note
%  \spass

\begin{aufgabe}[Von Hand laufen lassen und Eigenschaften \bestehen]\label{tue-first}\mbox{}
	\begin{enumerate}
		\item (\warmup) Füge die Schlüsselsequenz $7, 18, 2, 3, 14, 25, 1, 11, 12, 1332$ in eine Hashtabelle der Länge 11 ein, die verkettetes Hashing und die Hashfunktion $f(k) = k \mod 11$ benutzt.
		      \item\label{hand_sondieren} (\warmup) Füge die Schlüsselsequenz $2, 32, 43, 16, 77, 51, 1, 17, 42, 111$ in eine Hashtabelle der Länge 17 ein, die lineares Sondieren und die Hashfunktion $f(k) = k \mod 17$ benutzt.
		\item Lösche 111 und 51 aus der in~\ref{hand_sondieren} erzeugten Hashtabelle.
		      \item\label{wrong_delete} Angenommen, wir löschen ein Element~$x$ bei linearem Sondieren, ohne die Elemente im Cluster rechts von~$x$ wieder neu einzufügen.
		      Gib eine kürzestmögliche Sequenz von Wörterbuchoperationen an, sodass diese modifizierte Variante zu einem falschen Ergebnis führt.
		\item Sei $S$ eine Sequenz von Schlüsseln, die in einer Hashtabelle $A$ mittels verkettetem Hashing gespeichert sind.
		      Gegeben $A$, ist es möglich den größten Schlüssel aus $S$ effizient zu finden?
	\end{enumerate}
\end{aufgabe}

\begin{aufgabe}[Teiler in der Divisionsmethode \mittel]
	Bei der \emph{Divisionsmethode} wählen wir die Hashfunktion als $h(k)=k\bmod m$.
	\begin{enumerate}
		\item Betrachte die Hashfunktion $h(k) = k \bmod 10$ und die Schlüsselsequenz \[K = 0, 5, 20, 40, 65, 15, 90, 95, 80, 55\,.\] Warum ist diese Wahl der Hashfunktion problematisch für $K$?
		\item Erkläre, warum wir für~$m$ eine Primzahl bevorzugen.
	\end{enumerate}
\end{aufgabe}

\begin{aufgabe}[Faules Löschen bei linearem Sondieren \mittel]
	Die Methode aus~\ref{tue-first}\ref{wrong_delete} hat nicht funktioniert, wir versuchen es also nochmal anders.
	Wenn wir ein Element an Position $p$ löschen, dann hinterlassen wir jetzt eine Markierung, dass dort ein Element gelöscht worden ist.
	\begin{enumerate}
		\item Wie können \texttt{Search} und \texttt{Insert} modifiziert werden, damit diese Methode funktioniert?
		\item Welche Vor- und Nachteile hat diese Methode im Vergleich zu der Methode aus der Vorlesung hat?
	\end{enumerate}
\end{aufgabe}

\begin{aufgabe}[Spielserverstatistiken \mittel]
	Für dein neues, extrem erfolgreiches Onlinespiel \emph{Hashnite} willst du ermitteln, ob die vielen gespielten Spielsitzungen von einer kleinen Gruppe extrem aktiver Spieler:innen kommt oder aus einer großen Gruppe verschiedener Spieler:innen, die unregelmäßig spielen.
	Jede Spieler:in hat eine eindeutige ID, und von deinem Spieleserver aus kannst du auf die Liste aller IDs aus allen vergangenen Spielesitzungen zugreifen.
	\begin{enumerate}
		\item Entwirf einen Algorithmus, der die Anzahl an \emph{unterschiedlichen} Spieler:innen ermittelt, die jemals auf dem Server gespielt haben.
		\item Entwirf einen Algorithmus, der die Spieler:in ermittelt, die die meisten Spielsitzungen gespielt hat.
	\end{enumerate}
\end{aufgabe}

\begin{aufgabe}[Bitvektoren \schriftlich]
	Eine naive Implementierung würde einen \emph{Bitvektor} $B\in\{0,1\}^n$ als Array von \texttt{int}s darstellen, in dem jeder Eintrag~$0$ oder~$1$ ist:
	\begin{lstlisting}[language=C++,numbers=none]
		int[] B = new int[n]; // in C++
	\end{lstlisting}
	\vspace{-.5\baselineskip}
	Diese Darstellung verschwendet eine ganze Menge Platz, da für jedes Bit ein ganzer Integer genutzt wird, der 32 oder 64 Bit lang ist.%
	\footnote{%
		\texttt{bool} spart leider auch keinen Platz: \url{https://stackoverflow.com/questions/2064550/c-why-bool-is-8-bits-long}}
	%
	Wir wollen nun \emph{Bit-Operationen} nutzen, um Platz zu sparen.
	Angenommen wir arbeiten auf einem \emph{$w$-Bit Computer}, das heißt, die Register und Speicherzellen speichern jeweils $w$ Bits und primitive Datentypen wie Integer, Gleitkommazahlen und Zeiger werden mit $w$ Bits dargestellt.
	Viele Programmiersprachen unterstützen Bit-Manipulationen mit konstantem Zeitaufwand, wie \emph{bit shifts} (\verb$<<$ und \verb$>>$) und bitweise logische Operationen (\verb$|&^$).
	Löse die folgenden Teilaufgaben.
	\begin{enumerate}
		\item \bestehen %(\warmup{})
		      Für $w=8$, schreibe $2^4$, \verb$1<<4$, $2^8-1$,
		      $(2^7 \verb$^$ 2^4$) und
		      $(2^8-1)$ \verb$&$ $2^4$ in Binärdarstellung.
		\item \mittel Wir betrachten zunächst den Spezialfall $n=w$. Zeige, wie ein Bitvektor $B$ der Länge $w$ \textbf{kompakt} dargestellt werden kann, sodass das $i$-te Bit in konstanter Zeit ausgelesen oder geflippt werden kann. (Die schlimmste Laufzeit darf also nicht von $w$ abhängen.)
		\item \mittel Wir betrachten nun den allgemeinen Fall $n\ge w$. Wie kann ein Bitvektor $B$ der Länge $n$ \textbf{kompakt} dargestellt werden, sodass das $i$-te Bit in konstanter Zeit ausgelesen oder geflippt werden kann? (Die schlimmste Laufzeit darf also weder von $n$ noch von $w$ abhängen.)
		\item \mittel Entwirf eine Datenstruktur, die eine dynamische Menge $S\subseteq\{0,\dots,t\}$ darstellt und dabei die folgenden Operationen in konstanter Zeit unterstützt:
		      \begin{itemize}
			      \item \texttt{insert$(a)$} fügt der Menge $a$ hinzu, setzt also $S\gets S\cup \{a\}$.
			      \item \texttt{remove$(a)$} löscht $a$ aus der Menge, setzt also $S\gets S\setminus \{a\}$.
			      \item \texttt{has$(a)$} liefert $1$ wenn $a\in S$ und $0$ sonst.
		      \end{itemize}
	\end{enumerate}
\end{aufgabe}

\begin{aufgabe}[Sortieren in kleinen Universen \bestehen]%, \hard]
	Sei $A[1\hdots n]$ ein Feld von Zahlen aus ${\{1, \dots, n\}}$ sind. Entwirf einen Algorithmus, der $A$ in Zeit $O(n)$ sortiert.
\end{aufgabe}

\begin{aufgabe}[Nicht initialisierte Felder \note, \veryhard]
	Wir wollen eine abstrakte Datenstruktur implementieren, die sich so verhält wie ein Feld $A$ von ganzen Zahlen, das heißt, es werden die folgenden Operationen unterstützt:
	\begin{enumerate}
		\item $\texttt{init}(n,\texttt{default})$ initialisiert das Feld und legt den Integer \texttt{default} als Standardwert fest.
		\item $\texttt{set}(i,v)$ setzt den $i$-ten Eintrag auf den Integer~$v$.
		\item $\texttt{get}(i)$ liefert den $i$-ten Eintrag oder \texttt{default}, falls dieser noch nicht gesetzt wurde.
	\end{enumerate}
	Allerdings wird das Feld \emph{riesig} sein, deshalb wollen wir die Datenstruktur \emph{nicht} direkt als Feld implementieren, da wir keine Zeit darauf verschwenden wollen, alle Einträge auf \texttt{default} zu initialisieren.

	Entwirf eine Datenstruktur, die nur $O(n)$ Platz benötigt, Auslesen und Ändern in erwarteter konstanter Zeit pro Eintrag unterstützt und nur konstante Zeit für die Initialisierung benötigt.
%
	\emph{Hinweis: Hashtabellen und die Lösung zu Aufgabe 4.9 b) \enquote{Dynamische Felder} könnten hilfreich sein.}
\end{aufgabe}

% \newpage
% \begin{aufgabe}[Zufällige Hashfunktionen]
%   Oft benutzen wir nicht etwa feste Hashfunktionen, sondern zufällige.
%   Nur so können wir garantieren, dass die gewählte Hashfunktion die für Hashtabellen so wichtigen Eigenschaften auch wirklich haben, wie etwa die Annahme des einfachen gleichmäßigen Hashings.

%   Sei $p$ eine Primzahl.
%   Wir ziehen zwei zufällige Zahlen $a$ und $b$ gleichverteilt und unabhängig aus $\{0,\dots,p-1\}$.
%   Sei $h_{a,b}\colon\mathbb N\to\{0,\dots,p-1\}$ die Funktion mit $h_{a,b}(x)=ax+b\bmod p$.
%   % Betrachte die Prozedur:
%   % \begin{algorithmic}
%   %     \Procedure{Func}{$x_1,\dots,x_n$}
%   %     \State{Initialisiere ein Feld $T[0..p-1]$, in dem alle Einträge 0 sind.}
%   %     \State{Wähle zwei zufällige Zahlen $a$ und $b$ gleichverteilt und unabhängig aus $\{0,\dots,p-1\}.$}
%   %     \State{Sei $h:\mathbb N\to\{0,\dots,p-1\}$ die Funktion mit $h(x)=ax+b\bmod p$.}
%   %     \ForAll{$i\in\{1,\dots,n\}$}
%   %     \State $T[h(x_i)]\gets T[h(x_i)]+1$
%   %     \EndFor
%   %     \EndProcedure
%   % \end{algorithmic}

%   % Wir führen \Call{Func}{$x_1,\dots,x_n$} aus.
%   % Beweise:
%   \begin{enumerate}
%       \item \mittel Sei $x$ eine beliebige natürliche Zahl ungleich $0$ und sei $i$ ein beliebiges Element aus $\{0,\dots,p-1\}$.
%       Finde eine kurze Formel in Abhängigkeit von $p$ für die Wahrscheinlichkeit $\pr_{a,b}(h_{a,b}(x)=i)$, und beweise deine Antwort. (Mit anderen Worten, beweise dass die Annahme des einfachen gleichmäßigen Hashings für die zufällige Hashfunktion $h_{a,b}$ im Erwartungswert gilt.)
%       \item \note Seien $x,y\in\mathbb N$ zwei natürliche Zahlen mit $x\ne y$.
%       Beweise, dass \[\pr_{a,b}(h_{a,b}(x)=h_{a,b}(y))=\frac{1}{p^2}\] gilt. (Mit anderen Worten, beweise dass die Kollisionswahrscheinlichkeit $\frac{1}{p^2}$ ist.)
%       \item \note Sei $X\subset\mathbb N$ eine endliche Menge der Größe $|X|=k$. Sei $q_k$ die Wahrscheinlichkeit, dass \emph{keine} Kollision auftritt, das heißt:
%       \[q_k=\pr_{a,b}(\forall x,y\in X\colon x=y \vee h_{a,b}(x)\ne h_{a,b}(y))\,.\]
%       Beweise, dass folgende Ungleichungen gelten:
%       \[
%           \left(1-\frac{k}{p}\right)^k\le q_k \leq e^{-(k-1)k/(2p)}\,.
%       \]
%       (Das heißt, wir müssen bereits für $k=\Theta(\sqrt{p})$ mit Kollisionen rechnen.)\\
%       \emph{Hinweis: Die Ungleichung $(1-x)\le e^{-x}$ gilts stets und darf benutzt werden.}
%   \end{enumerate}

% %  \paragraph*{Hinweise zur Abgabe.}
% %  Bitte schreib deine Beweise möglichst kurz und elegant auf.
% %  Den \emoji{star} erhälst du für die vollständige und weitgehend korrekte Bearbeitung der Aufgabenteile~a) und~b). Aufgabenteil c) gibt einen Bonus in der Bewertung.

% \end{aufgabe}

\end{document}
