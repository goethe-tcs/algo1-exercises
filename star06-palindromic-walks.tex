% !TeX spellcheck = de_DE
\documentclass{uebung_cs}
\usepackage{algo121}
\blattname{\emoji{star}-Aufgabe: Palindromische Wege}

\usepackage{etoolbox}\AtBeginEnvironment{algorithmic}{\small}
\newcommand{\fett}[1]{\textbf{\boldmath\color{red!60!black}#1}}
%%%%%%%%%%%%%%%%%%%%%%%%%%%%%%%%%%%%%%%%%%%%%%%%%%%%%%%%%%%%%%%%%%%%%%%%%%%%

\begin{document}
\textit{\footnotesize For an English version of this exercise, see [\href{https://jeffe.cs.illinois.edu/teaching/algorithms/book/Algorithms-JeffE.pdf}{Erickson}, page 222]}.

Ein \emph{Palindrom} über dem Alphabet $\{\texttt{R},\texttt{B}\}$ ist eine Zeichenkette $s_1,s_2,s_3,\dots,s_n\in\{\texttt{R},\texttt{B}\}$, sodass $s_i=s_{n-i+1}$ für alle $i\in\{1,\dots,n\}$. Zum Beispiel sind \texttt{BRBRB} und \texttt{RBBR} Palindrome, aber \texttt{RBB} und \texttt{BRRR} nicht.

Sei $G$ ein beliebiger gerichteter Graph, in dem jede Kante entweder rot oder blau gefärbt ist, und seien $s,t$ zwei Knoten.
\begin{enumerate}
    \item Beschreibe einen Algorithmus, der entweder einen Weg von $s$ nach $t$ berechnet, für den die Sequenz von rot und blau entlang der Kanten des Weges ein Palindrom ist, oder korrekterweise feststellt, dass kein solcher Weg existiert.
    \item Beschreibe einen Algorithmus, der entweder einen \emph{kürzesten} Weg von $s$ nach $t$ berechnet unter allen Wegen, für die die Sequenz von rot und blau entlang der Kanten des Weges ein Palindrom ist, oder korrekterweise feststellt, dass kein solcher Weg existiert.
\end{enumerate}

\textbf{Hinweise zur Abgabe.}
Die Abgabe soll wie immer per PDF erfolgen und grobe Idee, Pseudocode (möglichst kurz und knapp!), Korrektheitsbeweis und Laufzeitanalyse umfassen. Um einen~\emoji{star} zu erhalten, muss Aufgabenteil~a) oder Aufgabenteil~b) zielgerichtet, nachvollziehbar, lesbar, vollständig, und korrekt gelöst sein.

\input{allgemeine-kriterien.inc}
\end{document}
