% !TeX spellcheck = de_DE
\documentclass{uebung_cs}
\usepackage{algo121}
\blattname{\emoji{star}-Aufgabe: Wiedereinfügen}

%%%%%%%%%%%%%%%%%%%%%%%%%%%%%%%%%%%%%%%%%%%%%%%%%%%%%%%%%%%%%%%%%%%%%%%%%%%%
\begin{document}

Sei $G$ ein stark zusammenhängender, gerichteter Graph mit nicht-negativen Kantengewichten, und seien $s$ und $t$ Knoten aus $G$.
Weiterhin sei $H$ ein Untergraph von $G$, der dieselbe Knotenmenge hat wie $G$ (also $V(H)=V(G)$), bei dem aber manche Kanten gelöscht wurden (also $E(H)\subsetneq E(G)$).

Wir wollen genau eine Kante aus $E(G)\setminus E(H)$ wieder in $H$ einfügen, sodass in dem neuen Graph die Länge des kürzesten Weges von $s$ nach $t$ so klein wie möglich ist (diese Länge könnte auch $\infty$ sein).
Beschreibe und analysiere einen Algorithmus, der für gegebenes $(G,H,s,t)$ die beste Kante zum Wiedereinfügen auswählt, und der in Zeit $O(m\log n)$ läuft.

\emph{Hinweis: Du kannst einen bereits bekannten Algorithmus mit geschickt gewählten Eingaben benutzen, oder du kannst einen bereits bekannten Algorithmus modifizieren. Meistens ist das Benutzen einfacher und eleganter!}

\input{allgemeine-kriterien.inc}
\end{document}
