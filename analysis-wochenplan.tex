% !TeX spellcheck = de_DE
\documentclass{uebung_cs}
\usepackage{algo121}
\blattname{Wochenplan: Analyse von Algorithmen}

%%%%%%%%%%%%%%%%%%%%%%%%%%%%%%%%%%%%%%%%%%%%%%%%%%%%%%%%%%%%%%%%%%%%%%%%%%%%


\begin{document}
\section*{Vorbereitung}
Lies CLRS Kapitel 3 sowie 4.3--4.5 und schau das Video der Woche.

\section*{Dienstag}
\begin{aufgabe}[Asymptotisches Wachstum, \warmup]\label{tue-first}\mbox{}
	Ordne die folgenden Funktionen in aufsteigender Reihenfolge nach ihrem asymptotischem Wachstum.
	Das heißt, $f(n)$ kommt vor $g(n)$, wenn $f(n) = o(g(n))$ gilt.
	\[n \log n\hspace*{15pt} n^2\hspace*{15pt} 2^n\hspace*{15pt} n^3\hspace*{15pt} \sqrt{n}\hspace*{15pt} n\]
	
\end{aufgabe}

\begin{aufgabe}[$\Theta$-Notation]
	Schreibe die folgenden Funktionen in $\Theta$-Notation.
	\begin{center}
		\begin{tabular}{ccc}
			$n^2 + n^3/2$
			&&
			$8\log_2^7 n + 34\log_2 n + \frac{1}{1000}n$\\
			$2^n + n^4$&&
			$2^n\cdot 7 + 5\log_2^3 n$\\
			$\log_2n + n\sqrt{n}$&&
			$n(n^2 - 18)\log_2 n$\\
			$n(n-6)$&&
			$n\log_2^4 n + n^2$\\
			$4\sqrt{n}$&&
			$n^3 \log_2 n + \sqrt{n}\log_2 n$\\
			$99\sin n + 99\cos n$&& $100+\frac{1}{n}$
		\end{tabular}
	\end{center}
	
\end{aufgabe}

\begin{aufgabe}[Looping Louie]
	Analysiere die Laufzeit für die folgenden Funktionen in $n$ und gib das Ergebnis in $O$-Notation an.
	\begin{center}
		\begin{tabular}{ccccc}

\begin{lstlisting}[language=Python]
Loop1(n)
i = 1
while i <= n do
	print "*"
	i = 2*i
return
\end{lstlisting}		
		
			&\mbox{}\hspace{2cm}\mbox{}&

\begin{lstlisting}[language=Python]
Loop2(n)
i = 1
while i <= n do
	print "*"
	i = 5*i
return
\end{lstlisting}

			
			&\mbox{}\hspace{2cm}\mbox{}&
			
\begin{lstlisting}[language=Python]
Loop3(n)
for i = 1 to n do
	j = 1
	while j <= n do
		print "*"
		j = j*2
return
\end{lstlisting}
		\end{tabular}
	\end{center}
\end{aufgabe}

\begin{aufgabe}[Asymptotische Aussagen]
	Sind die folgenden Aussagen wahr oder falsch?
	\begin{center}
		\begin{tabular}{ccc}
			$\frac{1}{20}n^2 + 100 n^3 = O(n^2)$
			&\mbox{}\hspace{2cm}\mbox{}&
			$\frac{n^3}{1000} + n + 100 = \Omega(n^2)$\\
			$\log_2 n + n = O(n)$&&
			$2^n + n^2 = \omega(n)$\\
			$2^{\log_2 n} = O(n)$&&
			$\log_4 n + \log_{16} n = \Theta(\log n)$\\
			$n^3(n-1)/5 = \Theta(n^3)$&&
			$n^{1/4} + n^2 = \Theta(n)$\\
			$\log_2^2n + n = \Theta(n)$&&
			$2^{\log_4n} = \Theta(\sqrt{n})$\\
			$n^{1.9}\log n = o(n^2/\log n)$&&
			$n\cdot(n\bmod 7) = \Theta(n)$\\
			$f(n)=\omega(g(n))$
			$\Longrightarrow$
			$f(n)=\Omega(g(n))$&&
			$f(n)=O(g(n))$
			$\Longrightarrow$
			$f(n)=o(g(n))$
		\end{tabular}
	\end{center}
\end{aufgabe}

\section*{Donnerstag}

\begin{aufgabe}[Verdopplungen]
	Löse die folgenden Teilaufgaben.
	\begin{enumerate}
		\item (\warmup) Algorithmus $A$ benötigt genau $7n^3$ Operationen für eine Eingabe der Länge $n$.
		Wie viele Operationen mehr benötigt $A$ bei doppelter Eingabelänge?
		\item Betrachte die Laufzeiten für einen Algorithmus $B$:
		\begin{table}[h!]
		\centering
		\begin{tabular}{lccccc}
			Eingabelänge (Bits) & 1000 & 2000 & 3000 & 4000 & 5000\\
			\midrule
			Dauer [Sekunden] & 5 & 20 & 45 & 80 & 125 \\
		\end{tabular}
		\end{table}
		
		Schätze die Laufzeit von B auf einer 6000 Bit langen Eingabe.
		Was ist vermutlich die asymptotische Laufzeit von B? Drück deine Vermutung in Abhängigkeit von Eingabelänge $n$ in $O$-Notation aus.
		\item Algorithmus $C$ benötigt für jede Verdoppelung der Eingabelänge 3 Sekunden länger. 
		Gib die asymptotischen Laufzeit von $C$ in Abhängigkeit von Eingabelänge $n$ in $O$-Notation an.
	\end{enumerate}
\end{aufgabe}

\begin{aufgabe}[Asymptotische Eigenschaften]
	Löse die folgenden Teilaufgaben.
	\begin{enumerate}
		\item Seien $f(n)$ und $g(n)$ asymptotisch nicht negative Funktionen.
		Beweise, dass folgendes gilt: $\max(f(n),g(n)) = \Theta(f(n) + g(n))$
		\item Erkläre warum die Aussage \enquote{Die Laufzeit von Algorithmus $D$ ist mindestens $O(n^2)$} keinen Sinn ergibt.
		\item Ist $2^{n+1} = O(2^n)$? Ist $2^{2n} = O(2^n)$? Beweise dies.
		\item Beweise, dass $\log_2(n!) = O(n \log n)$.
		\item (\hard) Beweise, dass $\log_2(n!) = \Omega(n\log n)$ gilt.
		Beweise, dass $\log_2(n!) = \Theta(n\log n)$ gilt.
	\end{enumerate}
\end{aufgabe}

\begin{aufgabe}[$k$-Merge Sort]
	Professorin M.\ Erge stellt ihren neuen Algorithmus, 3-Merge Sort, vor.
	3-Merge Sort funktioniert genau wie das uns bekannte Merge Sort, nur wird rekursiv in drei Teile anstatt zwei aufgeteilt, die daraufhin sortiert und wieder verflochten werden.
	Löse die folgenden Teilaufgaben.
	\begin{enumerate}
		\item Zeige, dass sich drei sortierte Felder in asymptotischer Linearzeit verflechten lassen.
		\item Führe eine Laufzeitanalyse für 3-Merge Sort durch.
		\item (\hard) Verallgemeinere den Algorithmus und die Analyse von 3-Merge Sort zu $k$-Merge Sort für $k>3$.\\
		Hat Prof.\ M.\ Erge den Durchbruch geschafft?
		Stellt $k$-Merge Sort eine Verbesserung gegenüber dem klassischen 2-Merge Sort dar?
	\end{enumerate}
\end{aufgabe}

\begin{aufgabe}[Maximale Teilfelder]
	Sei $A' \in \mathbb{Z}^n$ als ein Feld $A[0, \dots, n-1]$ gespeichert.
	Ein Teilfeld von $A$ ist ein genau dann ein \textit{maximales Teilfeld} $A[i..j]$ mit $0\leq i\leq j\leq n-1$, wenn die Summe $A[i] + A[i+1] + \cdots + A[j]$ maximal über alle möglichen Teilfelder ist.
	Löse die folgenden Aufgaben.
	\begin{enumerate}
		\item (\warmup) Schreib einen Algorithmus, der ein maximales Teilfeld von $A$ in Laufzeit $O(n^3)$ findet.
		\item Schreib einen Algorithmus, der ein maximales Teilfeld von $A$ in Laufzeit $O(n^2)$ findet.
		\item (\hard) Schreib einen \textit{Divide and Conquer} Algorithmus, der ein maximales Teilfeld von $A$ in Laufzeit $O(n\log n)$ findet.
		\item (\veryhard) Schreib einen Algorithmus, der einen maximales Teilfeld von $A$ in Laufzeit $O(n)$ findet.
	\end{enumerate}
\end{aufgabe}
\end{document}
