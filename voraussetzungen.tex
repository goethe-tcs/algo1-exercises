% LTeX: language=de-DE
\documentclass{uebung_cs}
\usepackage{algo123}
\blattname{Übungen zur Vorbereitung auf ALGO1}

%%%%%%%%%%%%%%%%%%%%%%%%%%%%%%%%%%%%%%%%%%%%%%%%%%%%%%%%%%%%%%%%%%%%%%%%%%%%
\begin{document}
Mit diesen Aufgaben kannst Du selbst prüfen, ob Du die nötigen Kenntnisse und Fähigkeiten besitzt, am Kurs \emph{Algorithmen und Datenstrukturen~1} teilzunehmen, oder ob Du einzelne Konzepte noch nacharbeiten musst.
Geh am besten wie folgt vor:
\begin{itemize}
    \item Lies Dir jede Aufgabe ganz aufmerksam durch. Wenn Du die Lösungsidee mental vor Dir siehst und Dir sicher bist, dass Du sie korrekt umsetzen kannst, brauchst Du nichts weiter zu tun.
    \item Wenn Du Dir nicht sicher sind, nimm Dir etwas Zeit und löse die Aufgabe schriftlich. Bei den Programmieraufgaben, schreib das Programm in einer Sprache Deiner Wahl (z.B.~Python) und lass es auch wirklich laufen, um es zu testen.
    \item Wenn Du mit mehreren Aufgaben Schwierigkeiten hast, musst Du vermutlich Deine mathematischen Grundkenntnisse oder Deine Programmierkenntnisse auffrischen. Hierzu empfiehlt sich der Vorsemesterkurs Informatik [\href{http://www-stud.informatik.uni-frankfurt.de/~lz_inf/Vorkurs/WS2021/webseite.html}{url}], oder ein kompakter Text zu den einführenden Aspekten des jeweiligen Themas, beispielsweise \emph{Introduction to Programming in Java} von Sedgewick und Wayne [\href{https://introcs.cs.princeton.edu/java/home/chapter1.pdf}{pdf}], oder das Vorlesungsskript \emph{Diskrete Modellierung} von Schnitger und Schweikardt [\href{http://algo.cs.uni-frankfurt.de/lehre/dismod/material/skript1920.pdf}{pdf}].
\end{itemize}

\begin{aufgabe}[Subtraktion]
    Schreib eine Funktion \texttt{sub}$(a,b)$, die zwei Zahlen $a$ und $b$ als Argument nimmt, und $a-b$ zurückgibt.
\end{aufgabe}

\begin{aufgabe}[Cut-off]
    Schreib eine Funktion \texttt{cut-off}$(A,k)$, die ein Array~$A$ von ganzen Zahlen und eine ganze Zahl $k$ als Argument nimmt und alle Einträge in $A$, die Wert größer als $k$ haben, auf $k$ setzt.
\end{aufgabe}

\begin{aufgabe}[Eingabe und Ausgabe]\mbox{}
    \begin{enumerate}
        \item Schreib ein Programm, das \texttt{Hallo Welt} auf die Konsole ausgibt.
        \item Schreib ein Programm, das als Kommandozeilenargument eine Zeichenkette $s$ liest und \texttt{Hallo $s$} auf die Konsole ausgibt.
        \item Schreib ein Programm \texttt{interaction}, das eine Zeichenkette $s$ von der Konsole liest (also von \texttt{stdin}) und \texttt{Hallo $s$} ausgibt.
    \end{enumerate}
\end{aufgabe}

\begin{aufgabe}[Summe]
    Schreib eine Funktion \texttt{sum}$(x)$, die eine positive ganze Zahl $x$ als Argument erwartet und die Summe $\sum_{i=1}^x i = 1+\cdots+x$ ausgibt.
\end{aufgabe}

\begin{aufgabe}[Sortiertheit prüfen]
    Schreib eine Funktion \texttt{inorder}$(a,b,c)$, die drei Zahlen $a,b,c$ als Argument erwartet und \texttt{true} zurückgibt, wenn die Werte streng monoton wachsend oder streng monoton fallend sind (das heißt, $a<b<c$ oder $a>b>c$), und ansonsten \texttt{false}.
\end{aufgabe}

\begin{aufgabe}[Elemente tauschen]\mbox{}
    \begin{enumerate}
        \item Schreib eine Funktion \texttt{swap}$(A,i,j)$, die ein Array $A$ und zwei Indizes $i$ und $j$ als Argument erwartet und die Werte an den Stellen $i$ und $j$ in $A$ vertauscht. Gib die Elemente von $A$ vor und nach dem Tausch aus.
        \item Schreib eine Funktion \texttt{reverse}$(A)$, die ein Array $A$ als Argument erwartet und die Reihenfolge der Elemente von $A$ umkehrt.
    \end{enumerate}
\end{aufgabe}

\begin{aufgabe}[Abstand]
    Schreib eine Funktion \texttt{dist}$(x,y)$, die zwei Zahlen $x$ und $y$ als Argument erwartet und den Euklidischen Abstand zwischen den Punkten $(0,0)$ und $(x,y)$ in der Ebene zurückgibt.
\end{aufgabe}

\begin{aufgabe}[Drei-Sortieren]
    Schreib eine Funktion \texttt{sort}$(a,b,c)$, die drei Zahlen $a,b,c$ nimmt und in aufsteigender Reihenfolge ausgibt. Benutze hierfür \texttt{Math.max()} und \texttt{Math.min()} (oder ähnliche Funktionen in der Programmiersprache Deiner Wahl).
\end{aufgabe}

\begin{aufgabe}[Fakultät]
    Schreib eine Funktion \texttt{fact}$(x)$, die eine Zahl $x$ als Eingabe nimmt und $x!=x\cdot (x-1)\cdot\ldots\cdot 1$ ausgibt.
    Die Funktion soll rekursiv sein, das heißt, die Funktion \texttt{fact} muss~$x!$ ausrechnen, indem sie sich selbst aufruft.
\end{aufgabe}

\begin{aufgabe}[Kopf oder Zahl]
    Schreib eine Funktion \texttt{flip}$(n,p)$, die eine Zahl $n$ und eine Wahrscheinlichkeit $p$ nimmt und $n$ unabhängige Münzwürfe simuliert, mit einer Münze, die mit Wahrscheinlichkeit $p$ Kopf zeigt. Gib die Sequenz der gefallenen Münzwürfe aus.
\end{aufgabe}

\begin{aufgabe}[Binärzahlen]
    Schreib eine Funktion \texttt{bin}$(x)$, die eine positive Zahl $x$ nimmt und ihre Binärdarstellung ausgibt.
\end{aufgabe}

\begin{aufgabe}[Matrixmultiplikation]
    Schreib eine Funktion \texttt{matrixMul}$(A,B)$, die zwei $n\times n$ Matrizen $A$ und $B$ als zweidimensionale Arrays nimmt und das Matrixprodukt $AB$ berechnet.
\end{aufgabe}

\begin{aufgabe}[Längste Hochebene]
    Schreib eine Funktion \texttt{plateau}$(A)$, die ein Array $A$ von ganzen Zahlen nimmt und die längste zusammenhängende Teilsequenz von gleichen Zahlen findet, wobei die Zahl direkt vorher und die Zahl direkt nachher kleiner sind.
\end{aufgabe}

\begin{aufgabe}[Potenzen und Logarithmen]\mbox{}
    \begin{enumerate}
        \item Vereinfache die folgenden Ausdrücke:
        $2^{10}$,\;
        $n^{10}\cdot n^{3}$,\;
        $\big(n^{10}\big)^{3}$,\;
        $n^{10} \cdot n^{-10}$,\;
        $n\cdot\sqrt{n}\cdot n^{-1/2}$,\;
        $\log_{10}(10000)$,\;
        $\log_2(2^n)$,\;
        $2^{\log_2 n}$,\;
        $\log_2(n^2)$,\;
        $\log_a(b^c)$.
        \item Welche der folgenden Identitäten ist im Allgemeinen richtig und welche falsch? (Überlege es Dir selbst, ohne irgendwo nachzulesen.)
        \begin{alignat*}{2}
            n^{a+b} &= n^a + n^b
            \qquad&\qquad
            n^{ab} &= n^a + n^b
            \\
            n^{a+b} &= n^a n^b
            \qquad&\qquad
            n^{ab} &= n^a n^b
            \\
            (a+b)^n &= a^n + b^n
            \qquad&\qquad
            (ab)^n &= a^n + b^n
            \\
            (a+b)^n &= a^n b^n
            \qquad&\qquad
            (ab)^n &= a^n b^n
            \\
            \log_a(x+y)&= \log_a x+ \log_a y
            \qquad&\qquad
            \log_a(xy)&= \log_a x+ \log_a y\\
            \log_a(x+y)&= \log_a x \cdot \log_a y
            \qquad&\qquad
            \log_a(xy)&= \log_a x\cdot \log_a y
        \end{alignat*}
    \end{enumerate}
\end{aufgabe}

\begin{aufgabe}[Summen]
    Gib geschlossene Ausdrücke an, und beweise die Gleichheit (z.B.~durch vollständige Induktion über~$n$):\quad
    \begin{enumerate*}
        \item $\sum_{i=1}^n i$,\;
        \item $\sum_{i=0}^n 2^{i}$,\;
        \item $\sum_{i=1}^\infty 2^{-i}$,\;
        \item $\sum_{i=0}^n \binom{n}{i}$
    \end{enumerate*}
\end{aufgabe}

\begin{aufgabe}[Grenzwerte]
    Welchen Wert haben die folgenden Grenzwerte? (Ohne Beweis.)
    \begin{enumerate*}
        \item $\lim_{n\to\infty} \tfrac1{n}$,\;
        \item $\lim_{n\to\infty} \tfrac1{\log_2 n}$,\;
        \item $\lim_{n\to\infty} \tfrac{n^{20}}{n^{19}}$,\;
        \item $\lim_{n\to\infty} \tfrac{\log_2 n}{\sqrt{n}}$,\;
        \item $\lim_{n\to\infty} n^{99} \cdot 2^{-n}$.
    \end{enumerate*}
\end{aufgabe}
\end{document}
