% !TeX spellcheck = de_DE
\documentclass{uebung_cs}
\usepackage{algo121}
\blattname{Wochenplan: Dynamische Programmierung}

\begin{document}
\section*{Vorbereitung}
Lies E Kapitel 3 ohne 3.6 und 3.9 und schau das Video der Woche.

\section*{Dienstag}
\begin{aufgabe}\mbox{}
    \begin{enumerate}
        \item(\warmup) Welche Editiersequenz ist hier visualisiert und wie viele Editieroperationen wurden benutzt?
        \begin{verbatim}
            P   L U   T   O
            P L A   N E T
        \end{verbatim}
        \item(\warmup) Fülle alle 25 Einträge der Tabelle \texttt{Edit[i,j]} mit $A=\texttt{PLUTO}$ und $B=\texttt{PLANET}$ aus. Gib die optimale Editiersequenz an.
        \item s
    \end{enumerate}
\end{aufgabe}

\begin{aufgabe}[Pascalsches Dreieck]\mbox{}\\
    Für die Binomialkoeffizienten gilt folgende Rekursionsformel:
    \begin{align*}
        \binom{n}{0}&=\binom{n}{n}=1&\quad\text{für alle $n\ge 0$, und}\\
        \binom{n}{k}&=\binom{n-1}{k-1}+\binom{n-1}{k}&\quad\text{für alle $n,k$ mit $ 1\le k \le n-1$.}
    \end{align*}
    Wir möchten eine Funktion implementieren, die $\binom{n}{k}$ ausrechnet.
    \begin{enumerate}
        \item Entwirf zunächst eine rekursive Funktion \Call{RecBinom}{$n,k$}, die schlicht die Rekursionsformel rekursiv benutzt. Wie viele rekursive Aufrufe braucht \Call{RecBinom}{$n,k$}? Gib die Antwort in asymptotischer Notation in Abhängigkeit von $n$ und $k$ an.
        \item Entwirf nun mit Hilfe von dynamischer Programmierung eine iterative Funktion \Call{IterBinom}{$n,k$}, die $O(nk)$ arithemtische Operationen ausführt und dabei $O(nk)$ Zahlen im Speicher hält.
        \item (\hard) Entwirf nun mit Hilfe von platzsparender dynamischer Programmierung eine iterative Funktion \Call{IterBinom2}{$n,k$}, die ebenfalls $O(nk)$ arithmetische Operationen braucht, dabei aber nur $O(k)$ Zahlen in Speicher halten muss.
    \end{enumerate}
\end{aufgabe}

\begin{aufgabe}[RNA]
    Die RNA eines RNA-Virus besteht aus den Molekülen Guanin, Adenin, Cytosin und Uracil, die in einer Kette angeordnet sind. Wir repräsentieren die RNA-Sequenz als eine Zeichenkette, die nur Zeichen aus dem Alphabet $\{\texttt{G}, \texttt{A}, \texttt{C}, \texttt{U}\}$ enthält, zum Beispiel \verb|AGACUAGUUAC|.
        Manchmal mutiert ein Virus und die RNA-Sequenz verändert sich leicht. Deine Aufgabe ist es nun, für zwei gegebene RNA-Sequenzen herauszufinden, wie ähnlich diese sich sind; damit kann man Rückschlüsse ziehen darauf, wie ähnlich sich zwei Virusvarianten sind. Aber nicht jede Änderung in der RNA ist gleich wahrscheinlich! Wir haben nun folgende (fiktive) Ähnlichkeitsmatrix $M[a,b]$ gegeben, die für jede Punktmutation (=Substitution) angibt, wie wahrscheinlich diese ist:

        \begin{tabular}{ccccc}
             & A  & G &  C & U\\
           A & 10 &-1 & -3 & -4\\
           G & -1 & 7 & -5 & -3\\
           C & -3 &-5 &  9 &  0\\
           U & -4 &-3 &  0 &  8\\
        \end{tabular}

        Je größer die Zahl ist, desto wahrscheinlich soll die entsprechende Substitution sein.
        Zum Beispiel hat eine Substitution von C auf G einen Ähnlichkeitwert von -5, aber ein korrektes Alignment von G mit G hat einen Ähnlichkeitswert von 7.
        Außerdem definieren wir noch einen Wert $d=-5$, der den Ähnlichkeitswert einer Einfügen oder Löschen Operation auf $-5$ setzt.
        Der Ähnlichkeitswert einer Editiersequenz ist die Summe der Ähnlichkeitswerte für jede Spalte.
        \begin{enumerate}
            \item(\warmup) Was ist der Ähnlichkeitswert dieser Editiersequenz?
            \begin{verbatim}
                G A U  
                C   C G
            \end{verbatim}
            \item Wie muss das dynamische Programm für die Edit Distance angepasst werden, um eine Editiersequenz zu berechnen, die einen möglichst großen Ähnlichkeitswert erzeugt?
        \end{enumerate}
\end{aufgabe}
\section*{Donnerstag}
\begin{aufgabe}[test]
\end{aufgabe}

\end{document}
