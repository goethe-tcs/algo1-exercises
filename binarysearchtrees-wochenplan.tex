% !TeX spellcheck = de_DE
\documentclass{uebung_cs}
\usepackage{algo121}
\blattname{Wochenplan: Disjunkte Mengen, Union-Find}

%%%%%%%%%%%%%%%%%%%%%%%%%%%%%%%%%%%%%%%%%%%%%%%%%%%%%%%%%%%%%%%%%%%%%%%%%%%%

\newboolean{programming}
\setboolean{programming}{false}

%%%%%%%%%%%%%%%%%%%%%%%%%%%%%%%%%%%%%%%%%%%%%%%%%%%%%%%%%%%%%%%%%%%%%%%%%%%%

\begin{document}
\section*{Vorbereitung}
Lies CLRS Kapitel 12 ohne 12.4 und schau das Video der Woche.

\section*{Dienstag}
\begin{aufgabe}[Binärbaumeigenschaften]\label{tue-first}
	\begin{enumerate}
		\item (\warmup) Welche der Bäume sind binäre Suchbäume?
		\item (\warmup) Wo in einem Baum befinden sich die Elemente mit dem kleinsten und größten Schlüssel?
		\item (\warmup) Gib die Reihenfolge, in der die Knoten traversiert werden für Präorder, Inorder und Postorder für Baum b) an.
		\item Verleich die Heap-Eigenschaft und die Suchbaum-Eigenschaft.
		\item Schreibe Pseudocode für eine iterative Variante der Knotentraversierung in Inorder.
		\item (\hard) Beweise oder widerlege.
		Wenn Knoten $v$ zwei Kinder hat, dann hat das Element mit dem nächstgrößeren Schlüssel kein linkes Kind und das Element mit dem nächstkleineren Schlüssel kein rechtes Kind.
	\end{enumerate}
\end{aufgabe}

\begin{aufgabe}[Blätter und Höhe]
	Sei $T$ ein Binärbaum mit $n$ Knoten und Wurzel $w$.
	\begin{enumerate}
		\item Entwirf einen rekursiven Algorithmus, der für Eingabe $w$ die Anzahl der Blätter in $T$ ausgibt.
		Schreibe deine Lösung in Pseudocode auf.
		\item Entwirf einen rekursiven Algorithmus, der für Eingabe $w$ die Höhe von $T$ ausgibt.
		Schreibe deine Lösung in Pseudocode auf.
		\ifthenelse{\boolean{programming}}{
		\item Implementiere deine Lösung in einer Sprache deiner Wahl.
		}{}
	\end{enumerate}
\end{aufgabe}

\end{document}
