% !TeX spellcheck = de_DE
\documentclass{uebung_cs}
\usepackage{algo121}
\blattname{\emoji{star}-Aufgabe: Rekursionen auf Bäumen}

%%%%%%%%%%%%%%%%%%%%%%%%%%%%%%%%%%%%%%%%%%%%%%%%%%%%%%%%%%%%%%%%%%%%%%%%%%%%
\begin{document}

Sei $T$ ein Binärbaum.
Jeder Knoten $x$ von $T$ hat die Eigenschaften $x.\texttt{parent}$, $x.\texttt{left}$ und $x.\texttt{right}$, welche auf den Elternknoten sowie auf das linke und rechte Kind von $x$ verweisen.
Wenn der Knoten keine Kinder hat (z.B. die Blätter) oder keinen Elternknoten hat (die Wurzel \texttt{root}), wird der jeweilige Wert auf \texttt{null} gesetzt.
Des Weiteren hat jeder Knoten $x$ eine Eigenschaft $x.\texttt{size}$, die auf einen Integer gesetzt werden kann.
Betrachte den folgenden Algorithmus.
\begin{algorithmic}
    \Procedure{Zero}{$x$}
    \If{$x\neq \texttt{null}$}
        \State{\Call{Zero}{$x.\texttt{left}$}}
        \State{\Call{Zero}{$x.\texttt{right}$}}
    \EndIf
    \EndProcedure
\end{algorithmic}
\begin{enumerate}
    \item Analysiere die Laufzeit von \textsc{Zero}$(x)$, wenn das Argument $x$ die Wurzel eines Teilbaums mit $n$ Knoten ist.
    \item Sei $T(x)$ der Teilbaum mit Wurzel $x$ und sei $|T(x)|$ die Anzahl an Knoten in $T(x)$.
    Entwirf einen rekursiven Algorithmus \textsc{InitSize$(x)$}, der für einen gegebenen Knoten $x$ und alle Knoten~$y$ im Teilbaum $T(x)$ die Variable $y.\texttt{size}$ auf $|T(y)|$ setzt.
    Schreib den Algorithmus in Pseudocode auf und analysiere die asymptotische Laufzeit als Funktion von $|T(x)|$.
    \item Wir bezeichnen eine Kante $(x,y)$ von Knoten $x$ zu einem seiner Kinder $y$ als \emph{rot} genau dann, wenn $|T(x)| \geq 2\cdot|T(y)|$.
    Entwirf einen rekursiven Algorithmus \textsc{RedEdge$(x)$}, der die Anzahl an roten Kanten im Teilbaum $T(x)$ berechnet.
    Schreib den Algorithmus in Pseudocode auf und analysiere die asymptotische Laufzeit als Funktion von $|T(x)|$.
    \item Wie viele rote Kanten treten in einem Pfad von der Wurzel zu einem Blatt auf?
    Gib asymptotische obere und untere Schranken an. Begründe deine Antwort.
\end{enumerate}
  
  \paragraph*{Hinweise zur Abgabe.}
  Um einen \emoji{star} zu erhalten, müssen alle Aufgabenteile vollständig und weitgehend korrekt bearbeitet werden.
\end{document}
