% !TeX spellcheck = de_DE
\documentclass{uebung_cs}
\usepackage{algo123}
\uebung{7}{}{}
\blattname{Prioritätswarteschlangen, Heaps (Woche 7)}

%%%%%%%%%%%%%%%%%%%%%%%%%%%%%%%%%%%%%%%%%%%%%%%%%%%%%%%%%%%%%%%%%%%%%%%%%%%%

\newboolean{programming}
\setboolean{programming}{false}

\tikzset{every tree node/.style={minimum width=20pt,draw,circle},
blank/.style={draw=none},
edge from parent/.style=
{draw, edge from parent path={(\tikzparentnode) -- (\tikzchildnode)}},
level distance=1.5cm}

\begin{document}
\textbf{Eigenständige Vorbereitung:}\\
Lies \emoji{book} CLRS Kapitel 6, sowie Appendix B.5 und schau dir das \emoji{television} Video der Woche an.

\textbf{Zeichenlegende:}
\legende{}

%  \schriftlich
%  \bestehen
%  \mittel
%  \note
%  \spass

\begin{aufgabe}[Heap-Eigenschaften]\label{tue-first}
	Löse die folgenden Teilaufgaben.
	\begin{enumerate}
		\item \bestehen %(\warmup)
    Welche der folgenden Bäume erfüllen die Heap-Eigenschaft?
		\begin{center}
			\begin{figure}[h]
				\begin{subfigure}[b]{0.3\textwidth}
					\hspace*{\fill}
					\scalebox{0.5}
					{
						\begin{tikzpicture}[sibling distance=20pt]
							\Tree
							[.16 
								[.13
									[.7
										\edge[]; {4}
										\edge[blank]; \node[blank]{};
									] 
									5
								]
								[.11
									9
									1
								]
							]
						\end{tikzpicture}
					}
					\hspace*{\fill}
					\caption{}
				\end{subfigure}
				\begin{subfigure}[b]{0.3\textwidth}
					\hspace*{\fill}
					\scalebox{0.5}
					{
						\begin{tikzpicture}[sibling distance=20pt]
							\Tree
							[.20 
								[.18 
									[.16
										13
										4
									]
									[.15
										7
										19
									]
								]
								[.14
									[.11
										\edge[]; {2}
										\edge[blank]; \node[blank]{};
									]
									10
								]
							]
						\end{tikzpicture}
					}
					\hspace*{\fill}
					\caption{}
				\end{subfigure}
				\begin{subfigure}[b]{0.3\textwidth}
					\hspace*{\fill}
					\scalebox{0.5}
					{
						\begin{tikzpicture}[sibling distance=20pt]
							\Tree
							[.9 
								[.8 
									[.6
										3
										2
									]
									\edge[blank]; \node[blank]{};
								]
								[.7
									[.5
										\edge[]; {1}
										\edge[blank]; \node[blank]{};
									]
									4
								]
							]
						\end{tikzpicture}
					}
					\hspace*{\fill}
					\caption{}
				\end{subfigure}
			\end{figure}
		\end{center}
				\item \bestehen %(\warmup)
        Welche der durch folgende Felder repräsentierten Bäume erfüllen die Heap-Eigenschaft?
		Index 0 wird nicht benutzt und ist deshalb mit -- markiert.
		\begin{center}
			$A = [\text{--},9,7,8,3,4]$\hspace*{3em}$B = [\text{--},12,4,7,1,2,10]$\hspace*{3em}$C = [\text{--},5,7,8,3]$
		\end{center}
		\item \bestehen %(\warmup)
    Sei $S = 4,8,11,5,21,\star,2,\star$ eine Sequenz von Operationen, wobei eine Zahl für das Einfügen dieser Zahl in den Heap steht und $\star$ für eine \texttt{ExtractMax} Operation.
		Wie sieht der Heap $H$ nach jeder einzelnen Operation aus, wenn $H$ anfangs leer ist?
		\item \bestehen Erfüllt ein sortiertes Feld die Heap-Eigenschaft?
		\item \bestehen Wo befindet sich in einem (Max-)Heap das kleinste Element?
		\item \mittel Zeige, dass \texttt{Insert}, \texttt{ExtractMax} und \texttt{IncreaseKey} die Heap-Eigenschaft aufrechterhalten.
    \item \note %(\hard)
    Angenommen wir erhalten $k$ sortierte Felder mit \textbf{insgesamt} $n$ Elementen als Eingabe.
		Zeige, wie sich alle Felder in Zeit $O(n\log k)$ zu einem einzelnen sortierten Feld der Länge~$n$ verflechten lassen.
	\end{enumerate}
\end{aufgabe}

\begin{aufgabe}[Priogruppen-Politik \bestehen]%, \warmup]
	Die Kakistokratische Partei will deine Hilfe, um ihre neue \enquote{Frischluft}-Politik umzusetzen.
	Entwirf ein Bürgerregister, das alle Bürger:innen und ihre Gehälter so speichert, dass man die Person mit dem geringsten Einkommen möglichst schnell finden und ausbürgern kann.

	Das System soll die folgenden Operationen unterstützen:
	\begin{itemize}
		\item \texttt{Insert(c,i)} fügt eine Person mit der Sozialversicherungsnummer $c$ und dem jährlichen Gehalt $i$ ein.
		\item \texttt{DeportLowestIncome()} Gibt die Person mit dem niedrigsten Einkommen aus und entfernt sie aus dem System.
	\end{itemize}
	Entwirf eine möglichst effiziente Datenstruktur, die das System implementiert.
\end{aufgabe}

\begin{aufgabe}[Operationen für Prioritätswarteschlangen]
	Wir wollen nun die Menge an zur Verfügung stehenden Operationen für Prioritätswarteschlangen vergrößern.
	Wir interessieren uns hierbei für die folgenden Operationen:
	\begin{itemize}
		\item \texttt{RemoveLargest($m$)} entfernt das $m$t-größte Element der Prioritätswarteschlange.
		\item \texttt{Delete($x$)} entfernt Element $x$ aus der Prioritätswarteschlange.
		\item \texttt{Fusion($x,y$)} entfernt Elemente $x$ und $y$ aus der Prioritätswarteschlange und fügt ein neues Element~$z$ mit Schlüssel $x$.key + $y$.key ein.
		\item \texttt{FindLarger($k$)} gibt all jene Elemente der Prioritätswarteschlange aus, deren Schlüssel mindestens so groß wie $k$ ist.
		\item \texttt{ExtractMin()} gibt das Element der Prioritätswarteschlange mit dem kleinsten Schlüssel aus und entfernt es.
	\end{itemize}
	Wir wollen diese Operationen effizient implementieren, ohne dass sich die Komplexität der Standardoperationen \texttt{Insert}, \texttt{IncreaseKey}, \texttt{Max} und \texttt{ExtractMax} ändert.

	Sei $n$ die Anzahl der Elemente in der Prioritätswarteschlange.
	Löse die folgenden Teilaufgaben:
	\begin{enumerate}
		\item \mittel Erkläre wie sich \texttt{RemoveLargest($m$)} mit Zeitbedarf $O(m\log n)$ implementieren lässt.
		\item \mittel Erkläre wie sich \texttt{Delete($x$)} und \texttt{Fusion($x,y$)} mit Zeitbedarf $O(\log n)$ implementieren lässt.
		\item \mittel %(\hard)
    Erkläre wie sich \texttt{ExtractMin()} mit Zeitbedarf $O(\log n)$ implementieren lässt.
		\item \mittel %(\hard)
    Erkläre wie sich \texttt{FindLarger($k$)} mit Zeitbedarf $O(m)$ implementieren lässt, wobei $m$ die Anzahl der Elemente mit Schlüssel $\geq k$ ist.
	\end{enumerate}
\end{aufgabe}

\begin{aufgabe}[Zusätzliche Daten \mittel]
	Sei $A[0..n-1]$ ein als Feld gespeicherter Heap.
	Jedes Element~\texttt{x} in dem Heap wird durch einen Index $i$ repräsentiert und hat einen Schlüssel \texttt{x.key}, der als $A[i]$ gespeichert ist.
	Es ist oftmals nützlich, zusätzliche Daten \texttt{x.data} zu speichern, die mit einem Element \texttt{x} assoziiert sind.
	Modifiziere die Datenstruktur so, dass eine neue Operation \texttt{Data$(i)$} in Zeit~$O(1)$ die zusätzlichen Daten des Elements mit Index $i$ zurückliefert. Hierbei dürfen sich die asymptotischen Laufzeiten der Standardoperationen des Heaps nicht verändern.
\end{aufgabe}

\begin{aufgabe}[Eigenschaften von Heaps \mittel]
	Sei $T = (V,E)$ ein vollständiger Binärbaum von Höhe~$h$.
	Löse die folgenden Teilaufgaben.
	\begin{enumerate}
		\item Zeige, dass für die Anzahl an Knoten $|V| = 2^{h+1} - 1$ gilt.\\
		\textit{Hinweis: Begründe, dass $|V| = 1 + 2 + 4 + \hdots + 2^h$ gilt und betrachte diesen Wert als Binärzahl.}
		\item Zeige: Für die Summe $S$ mit $S = n / 4 \cdot 1 + n / 8 \cdot 2 + n / 16 \cdot 3 + n / 32 \cdot 4 + \hdots$ gilt $S = \Theta(n)$.\\
		\textit{Hinweis: Berechne $S - S/2$}
	\end{enumerate}
\end{aufgabe}

\ifthenelse{\boolean{programming}}{
\begin{aufgabe}[Implementierung von Heaps]
	Wir interessieren uns dafür eine Prioritätswarteschlange mittels eines Heap auf einem Feld zu implementieren.
	\begin{enumerate}
		\item Implementiere die zuvor genannte Datenstruktur mitsamt \texttt{Insert} und \texttt{ExtractMax} Operationen in einer Programmiersprache deiner Wahl.
	\end{enumerate}
\end{aufgabe}
}{}

\begin{aufgabe}[Summen]
	Sei $A[0..n-1]$ ein Feld von ganzen Zahlen.
	Wir interessieren uns für die folgenden Operationen:
	\begin{itemize}
		\item \texttt{Sum($i,j$)} gibt $A[i] + A[i+1] + \hdots + A[j]$ aus.
		\item \texttt{Change($i,x$)} setzt $A[i]$ auf den Wert $x$.
	\end{itemize}
	Löse die folgenden Teilaufgaben:
	\begin{enumerate}
		\item \bestehen %(\warmup)
    Entwirf eine Datenstruktur, die \texttt{Sum} mit $O(1)$ Zeit und $O(n^2)$ Platz unterstützt.
		\item \mittel %(\hard)
    Entwirf eine Datenstruktur, die \texttt{Sum} mit $O(1)$ Zeit und $O(n)$ Platz unterstützt.
		\item \note (\veryhard) Entwirf eine Datenstruktur, die \texttt{Sum} und \texttt{Change} beide mit $O(\log n)$ Zeit und $O(n)$ Platz unterstützt.
	\end{enumerate}
\end{aufgabe}

\begin{aufgabe}[Sitze in einem Parlament \schriftlich\mittel]
  Schreibe ein Programm in C/C\texttt{++}, Java, Python oder einer anderen gängigen Programmiersprache (kein Pseudocode), welches das folgende Problem löst: 
  Verteile die $m$ Sitze in einem Parlament nach einer Wahl auf $n$ Parteien.\\[0.25cm]
  Die Platzvergabe verläuft nach dem \emph{D'Hondt-Verfahren}:
  für $i \in \{1, \dots, n\}$ bezeichne $v_i \in \mathbb N$ die Anzahl der Stimmen für Partei $i$.
  Für jede Partei $i$ wird ein Quotient $q_i$ berechnet, welcher anfangs auf $q_i := v_i/1$ gesetzt wird.
  Hat Partei $j$ den größten Quotienten, wird ihr ein Sitz zugeteilt.
  Anschließend wird ihr Quotient folgendermaßen aktualisiert:
  \begin{align*}
      q_j := \frac{v_j}{s_j + 1},
  \end{align*}
  wobei $s_j$ die Anzahl der Sitze, welche bisher Partei $j$ zugeordnet wurden, bezeichnet.
  Anfangs wird die Anzahl der zugeordneten Sitze für alle Parteien auf $0$ gesetzt.
  Dieser Vorgang wird wiederholt, bis alle $m$ verfügbaren Sitze vergeben sind.
  
  \textbf{Eingabe.}
  Die Datei besteht aus mehreren Zeilen.
  In der ersten Zeile sind $n$ und $m$ durch ein Leerzeichen getrennt gegeben.
  Beide Zahlen sind in der Menge $\{1, \dots, 2\,000\,000\}$ enthalten.
  In der $i$-ten der $n$ darauffolgenden Zeilen ist die ganze Zahl $v_i$ gegeben.
  Jede Partei erhält mindestens eine Stimme.
  Die Anzahl aller Stimmen ist mindestens so groß wie die Anzahl der zu verteilenden Sitze, es gilt also $v_1 + \dots + v_n \geq m$.
  Unsere Eingaben sind so konstruiert, dass der letzte Sitz eindeutig zugeordnet werden kann -- es muss keine Logik eingebaut werden, welche bei Gleichstand entscheidet.
  
  \textbf{Ausgabe.}
  Die Sitzverteilung, wobei sich die Reihenfolge in der Ausgabe mit der Reihenfolge in der Eingabe vertragen soll. Betrachte hierzu die Beispiele.
  
  \textbf{Beispiele.}\\
	\begin{tabular}{p{0.2\textwidth}p{0.2\textwidth}p{0.2\textwidth}p{0.2\textwidth}}
	\texttt{\bfseries 1.in}
	\begin{verbatim}
	2 2
	10
	10000000
	\end{verbatim}
	&
	\texttt{\bfseries 1.ans}
	\begin{verbatim}
	0
	2
	\end{verbatim}
	&
	\texttt{\bfseries 2.in}
	\begin{verbatim}
	2 3
	12
	11
	\end{verbatim}
	&
	\texttt{\bfseries 2.ans}
	\begin{verbatim}
	2
	1
	\end{verbatim}
	\\[\baselineskip]
	\texttt{\bfseries 3.in}
	\begin{verbatim}
	2 4
	12
	11
	\end{verbatim}
	&
	\texttt{\bfseries 3.ans}
	\begin{verbatim}
	2
	2
	\end{verbatim}
	&
	\texttt{\bfseries 4.in}
	\begin{verbatim}
	2 4
	17
	10
	\end{verbatim}
	&
	\texttt{\bfseries 4.ans}
	\begin{verbatim}
	3
	1
	\end{verbatim}
	\end{tabular}
  
  \textbf{Erklärung zu Beispiel 4.}
  Es traten 2 Parteien zur Wahl an, und insgesamt sollen 4 Sitze vergeben werden.
  Partei 1 erhält den ersten Sitz, da sie die meisten Stimmen erhalten hat. 
  Anschließend wird ihr Quotient auf $q_1 = \frac{17}{2}$ gesetzt.
  Partei 2 erhält den nächsten Sitz, denn $10 > \frac{17}{2}$.
  Anschließend wird ihr Quotient auf $q_2 = \frac{10}{2} = 5$ gesetzt.
  Der dritte Sitz geht an Partei 1, denn es gilt $q_1 = \frac{17}{2} > 5 = q_1$.
  Der Quotient von Partei~1 wird anschließend auf $q_1 = \frac{17}{3}$ gesetzt.
  Auch der letzte Sitz geht an Partei 1, denn es gilt $q_1 = \frac{17}{3} > 5 = q_2$.
  Insgesamt erhält Partei 1 also drei Sitze und Partei 2 erhält einen.
  
  \begin{tabular}{p{0.3\textwidth}p{0.3\textwidth}}
  \texttt{5.in}
  \begin{verbatim}
  4 14
  38
  35
  36
  37
  \end{verbatim}
  &
  \texttt{5.ans}
  \begin{verbatim}
  4
  3
  3
  4
  \end{verbatim}
  \end{tabular}
  
  \textbf{Größere Beispiele.}
  Teste dein Programm! Hier sind größere Beispiele:
  \url{https://files.tcs.uni-frankfurt.de/algo1/seatallocation-tests.zip}.  
  Stell sicher, dass dein Programm für alle Eingaben \texttt{X.in} \emph{exakt} die entsprechende Ausgabedatei \texttt{X.ans} erzeugt.
  
  \textbf{Tipps.}
  Achte auf Rundungsfehler und möglichen Überlauf von \texttt{int}s.
	Es ist keine Überraschung, dass diese Aufgabe mit einer Prioritätswarteschlange gelöst werden soll.
  
  \textbf{Hinweise zur Abgabe.}
  Die Datei \texttt{v2-015-secret.ans} fehlt.
  Deine Abgabe soll die SHA1-Summe der korrekten Datei enthalten.
  Zum Beispiel erhält man die SHA1-Summe der Datei \texttt{v2-014-19.ans} wie folgt:
  \begin{verbatim}
  $ sha1sum v2-014-19.ans 
  22d43373a4104696005cb3db1fa8f3f0c873090a  v2-014-19.ans
  \end{verbatim}
  Deine Abgabe soll wie immer per PDF erfolgen und die grobe Idee, den diesmal echten Code, den Korrektheitsbeweis, die Laufzeitanalyse, und die SHA1-Summe von \texttt{v2-015-secret.ans} enthalten.
  Weiterhin zu beachten:
  \begin{itemize}
  \item Der Code darf maximal \textbf{60} Zeilen lang sein (jede Zeile mit maximal 100 Zeichen). Möglichst kurz und elegant! (Eine 20-Zeilen Lösung in Python ist möglich. Kommentare zählen nicht dazu.)
  \item In Python kann man einen Min-Heap mit dem Modul \texttt{heapq} nutzen, siehe z.B. hier: \href{https://www.geeksforgeeks.org/heap-queue-or-heapq-in-python/}{geeksforgeeks.org/heap-queue-or-heapq-in-python/}. Um ein Element \texttt{x} mit Priorität $7$ auf einen heap \texttt{heap} einzufügen, verwende beispielsweise \texttt{heapq.heappush((7,x), heap)}. Um einen Min-Heap zu einem Max-Heap zu machen, kann man die Vorzeichen der Prioritäten ändern. Falls du deine eigene Implementierung der Prioritätswarteschlange nutzen möchten, zählt diese Implementierung nicht zum Zeilenlimit dazu. Wichtig: Die Datenstruktur darf nur so benutzt werden, wie die in der Vorlesung beschriebene abstrakte Datenstruktur das erlaubt.
  \end{itemize}

\end{aufgabe}

\end{document}
