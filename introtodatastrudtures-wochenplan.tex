% !TeX spellcheck = de_DE
\documentclass{uebung_cs}
\usepackage{algo121}
\usepackage{algorithmic}
\blattname{Wochenplan: Stapel, Warteschlangen, Verkettete Listen und Bäume}

%%%%%%%%%%%%%%%%%%%%%%%%%%%%%%%%%%%%%%%%%%%%%%%%%%%%%%%%%%%%%%%%%%%%%%%%%%%%


\begin{document}

\section*{Vorbereitung}
Lies CLRS Einleitung von Teil III und Kapitel 10, Kapitel 17.4 bis Mitte von 17.4.1.

\section*{Dienstag}

\begin{aufgabe}[Stapel und Warteschlangen]
	Löse die Teilaufgaben.
	\begin{enumerate}
		\item (\warmup) Wie sieht die Abfolge \texttt{PUSH(4), PUSH(1), PUSH(3), POP(), PUSH(8), POP()} auf einem initial leerem Stapel, der auf einem Feld der Länge 6 implementiert wurde, aus?
		\item Zeige wie zwei Stapel $S_1, S_2$ auf einem Feld $A$ der Länge $N$ implementierbar ist.
		Hierbei soll es zu keinem Stapel Overflow kommen, es sei denn die Anzahl der Elemente in $S_1$ und $S_2$ ist größer als $N$.
		Die \texttt{PUSH} und \texttt{POP} Operationen sollen $O(1)$ Zeit benötigen.
		\item (\warmup) Wie sieht die Abfolge \texttt{ENQUEUE(4), ENQUEUE(1), ENQUEUE(3), DEQUEUE(), ENQUEUE(8), DEQUEUE()} auf einer initial leeren Warteschlange, die auf einem Feld der Länge 6 implementiert wurde, aus?
		\item (\hard) Zeige, wie eine Warteschlange $Q$ und ihre Operationen mit nur zwei Stapel $S_1, S_2$ implementierbar ist.
		Gib eine Laufzeitanalyse für die Operationen an.
	\end{enumerate}
\end{aufgabe}

\begin{aufgabe}[Nahezu eine ehemalige dänische Klausuraufgabe]
	Sei $S$ ein Stapel.
	Führe die folgenden Operationen von links nach rechts aus: Ein Buchstabe $i$ steht hierbei für $S.$\texttt{PUSH(}$i$\texttt{)} und * steht für $S.$\texttt{POP()}.
	\[IFI*G**OE*THEU**NI\]
	Gib die Sequenz der Buchstaben, die durch die Pop-Aufrufe ausgegeben werden, entsprechend der gegebenen an.
\end{aufgabe}

\begin{aufgabe}[Algorithmen auf verkettete Listen]
	Betrachte die Algorithmen \texttt{FOO} und \texttt{BAR} und die verkettete Liste darunter.
	\begin{center}
	\begin{minipage}{0.45\textwidth}
		\begin{algorithmic}
			\STATE \texttt{FOO}$(head)$
			\STATE $x = head$
			\STATE $c = 0$
			\WHILE{$x \neq null$}
				\STATE $x = x.next$
				\STATE $c = c + 1$
			\ENDWHILE
			\RETURN c
		\end{algorithmic}
	\end{minipage}%
		\hfill
	\begin{minipage}{0.45\textwidth}
		\begin{algorithmic}
			\STATE \texttt{BAR}$(x,s)$
			\IF{$x == null$}
				\RETURN s
			\ELSE
				\RETURN \texttt{Bar}$(x.next)$
			\ENDIF
		\end{algorithmic}
	\end{minipage}\\%
	\end{center}
	
	\textbf{!!!Hier kommt eine verkettete Liste rein!!!}

	\begin{enumerate}
		\item (\warmup) Führe \texttt{FOO}(head) händisch durch.
		\item (\warmup) Erkläre was \texttt{FOO} berechnet.
		\item Führe \texttt{BAR}(head,0) händisch durch.
		\item Erkläre was \texttt{BAR} berechnet.
	\end{enumerate}
\end{aufgabe}

\begin{aufgabe}[Implememtierung verketteter Listen, \warmup]
	Sei $x$ ein Element in einer einfach verketteten Liste wie in den Folien beschrieben.
	Löse die folgenden Teilaufgaben.
	\begin{enumerate}
		\item Sei $x$ nicht das letzte Element in der Liste.
		Worin resultiert der folgende Code-Schnipsel?
		\hspace*{20pt}\texttt{x.next = x.next.next;}

		\item Sei $t$ ein neues Element, das noch nicht in der Liste enthalten ist.
		Worin resultiert der folgende Code-Schnipsel?
		\hspace*{20pt}\texttt{t.next = x.next;}
		\hspace*{20pt}\texttt{x.next = t;}

		\item Voraussetzungen wie in b).
		Bringt der folgende Code-Schnipsel die gleichen Resultate?
		Wenn nein was sonst?
		\hspace*{20pt}\texttt{x.next = t;}
		\hspace*{20pt}\texttt{t.next = x.next;}
	\end{enumerate}
\end{aufgabe}

\begin{aufgabe}[Implememtierung von Stapeln und Warteschlangen]
	Implementiere die folgenden Datenstrukturen in einer Programmiersprache deiner Wahl.
	\begin{enumerate}
		\item Ein Stapel für Integer mittels einer einfach verketteten Liste.
		\item Eine Warteschlange für Integer mittels einer einfach verketteten Liste.
	\end{enumerate}
\end{aufgabe}

\begin{aufgabe}[Sortierte verkettete Listen]
	Sei $L$ eine einfach verkettete Liste von $n$ Integern in sortierter Reihenfolge.
	Löse die folgenden Aufgaben.
	\begin{enumerate}
		\item Entwirf einen Algorithmus, der einen neuen Integer in $L$ einfügt, sodass die Liste danach noch sortiert ist.
		\item Ein Freund schlägt vor binäre Suche zu verwenden um das Suchen in sortierten verketteten Listen zu beschleunigen. Wird das funktioniere? Begründe.
	\end{enumerate}
\end{aufgabe}

\begin{aufgabe}[Eine Liste umkehren]
	Entwirf einen Algorithmus, der eine verkettete Liste $L_{In}$ mit $n$ Elementen als Eingabe erhält und eine verkette Liste $L_{Out}$ ausgibt, die die Elemente $L_{In}$ in umgekehrter Reihenfolge enthält.
	Der Algorithmus soll $O(n)$ Zeit benötigen und dabei, zusätzlich zur Ausgabe, nicht mehr als konstant viel zusätzlichen Speicherplatz benötigen\footnote{Zusätzlicher Speicherplatz hängt in keiner Weise von $n$ ab und wächst nicht mit einer längeren Eingabe.}.
\end{aufgabe}

\begin{aufgabe}
	32 Inhaftierte sind zu lebenslänglicher Einzelhaft verurteilt.
	Der Gefängnisdirektor schlägt den Inhaftierten einen zwar nicht mit Menschenrechten, aber mit einer algorithmischen Fragestellung vereinbaren, Deal vor.
	Der Direktor hat eine Schüssel voll mit den Zellennummern aller Inhaftierten.
	Jeden Tag zieht der Direktor eine Nummer lässt die Inhaftierte, aus der korrespondierenden Zelle, in das Vernehmungszimmer des Gefängnisses und packt die Zellennummer zurück in die Schüssel.
	Das Vernehmungszimmer ist leer, abgesehen von $k$ Lichtschaltern.
	Aktionen, die die Inhaftierte durchführen ist ist die Schalter an und ausschalten.
	Zudem kann sich die Inhaftierte dazu entscheiden dem Direktor zu sagen, dass sie im überzeugt ist, dass alle 32 Inhaftierte mindestens einmal im Zimmer gewesen sind.
	Liegt sie in dieser Annahme richtig werden alle Inhaftierten freigelassen.
	Liegt sie in dieser Annahme falsch werden alle Inhaftierten hingerichtet.
	Die Inhaftierten bekommen ein Treffen um sich eine Strategie zu überlegen, bevor sie wieder voneinander isoliert werden.
	Anfangs sind alle Lichtschalter ausgeschaltet.
	\begin{enumerate}
		\item Ist es möglich für $k=32$ eine Strategie zu entwickeln, die mit 100\%iger Sicherheit funktioniert?
		\item Für $k=5$?
		\item (\veryhard) Für $k=1$?
	\end{enumerate}
\end{aufgabe}

\end{document}
