% !TeX spellcheck = de_DE
\documentclass{uebung_cs}
\usepackage{algo121}
\blattname{\emoji{star}-Aufgabe: Zufällige Hashfunktionen}

%%%%%%%%%%%%%%%%%%%%%%%%%%%%%%%%%%%%%%%%%%%%%%%%%%%%%%%%%%%%%%%%%%%%%%%%%%%%
\begin{document}
Oft benutzen wir nicht etwa feste Hashfunktionen, sondern zufällige.
Nur so können wir garantieren, dass die gewählte Hashfunktion die für Hashtabellen so wichtigen Eigenschaften auch wirklich haben, wie etwa die Annahme des einfachen gleichmäßigen Hashings.

Sei $p$ eine Primzahl.
Wir ziehen eine zufällige Zahl $a$ gleichverteilt aus $\{1,\dots,p-1\}$ und unabhängig eine zufällige Zahl $b$ gleichverteilt aus $\{0,\dots,p-1\}$.
Sei $h_{a,b}\colon\mathbb N\to\{0,\dots,p-1\}$ die Funktion mit $h_{a,b}(x)=ax+b\bmod p$.
% Betrachte die Prozedur:
% \begin{algorithmic}
%     \Procedure{Func}{$x_1,\dots,x_n$}
%     \State{Initialisiere ein Feld $T[0..p-1]$, in dem alle Einträge 0 sind.}
%     \State{Wähle zwei zufällige Zahlen $a$ und $b$ gleichverteilt und unabhängig aus $\{0,\dots,p-1\}.$}
%     \State{Sei $h:\mathbb N\to\{0,\dots,p-1\}$ die Funktion mit $h(x)=ax+b\bmod p$.}
%     \ForAll{$i\in\{1,\dots,n\}$}
%     \State $T[h(x_i)]\gets T[h(x_i)]+1$
%     \EndFor
%     \EndProcedure
% \end{algorithmic}

% Wir führen \Call{Func}{$x_1,\dots,x_n$} aus.
% Beweise:
\begin{enumerate}
    \item Sei $x$ eine beliebige natürliche Zahl ungleich $0$ und sei $i$ ein beliebiges Element aus $\{0,\dots,p-1\}$.
    Finde eine kurze Formel in Abhängigkeit von $p$ für die Wahrscheinlichkeit $\pr_{a,b}(h_{a,b}(x)=i)$, und beweise deine Antwort. (Mit anderen Worten, beweise, dass die Annahme des einfachen gleichmäßigen Hashings für die zufällige Hashfunktion $h_{a,b}$ im Erwartungswert gilt.)
    \item Sei $m\in\mathbb N$ mit $m<p$.
    Als eigentliche Hashfunktion $h'_{a,b}\colon\{0,\dots,p-1\}\to\{0,\dots,m\}$ definieren wir nun $h'_{a,b}(x)=(h_{a,b}(x) \bmod m)$.
    Beweise, dass für alle natürlichen $x,y$ mit $x<y<p$ gilt:
    \[\pr_{a,b}(h'_{a,b}(x)=h'_{a,b}(y))=\frac{1}{p-1}\,.\]
    (Mit anderen Worten, beweise, dass die Kollisionswahrscheinlichkeit $\frac{1}{p-1}$ ist.)
    \item Sei $0\le k\le m$. Wir ziehen $k$ zufällige Elemente $z_1,\dots,z_k\in \{0,\dots,m-1\}$ unabhängig und gleichverteilt aus $\{0,\dots,m-1\}$ (``mit Zurücklegen''). Sei $q_k$ die Wahrscheinlichkeit, dass \emph{keine} Kollision auftritt, das heißt:
    \[q_k=\pr_{z_1,\dots,z_k}\Big(\forall i,j\in \{1,\dots,k\}\colon i\ne j \implies z_i\ne z_j\Big)\,.\]
    Beweise, dass folgende Ungleichungen gelten:
    \[
        \left(1-\frac{k}{m}\right)^k\le q_k \leq e^{-(k-1)k/(2m)}\,.
    \]
    (Das heißt, selbst in dem idealen Fall, dass $k$ Schlüssel \emph{perfekt zufällig} auf $m$ Buckets verteilt werden, müssen wir bereits für $k=\Theta(\sqrt{m})$ mit mindestens einer Kollision rechnen.)\\
    \emph{Hinweis: Die Ungleichung $(1-x)\le e^{-x}$ gilts stets und darf benutzt werden.}
\end{enumerate}

\paragraph*{Hinweise zur Abgabe.}
Bitte schreib deine Beweise möglichst kurz und elegant auf.
Den \emoji{star} erhälst du für die vollständige und korrekte Bearbeitung von Aufgabenteil~a). Aufgabenteile b) und c) geben einen Bonus in der Bewertung.

\end{document}
