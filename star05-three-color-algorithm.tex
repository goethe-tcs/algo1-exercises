% !TeX spellcheck = de_DE
\documentclass{uebung_cs}
\usepackage{algo121}
\blattname{\emoji{star}-Aufgabe: 3-Farben Algorithmus}

\usepackage{etoolbox}\AtBeginEnvironment{algorithmic}{\small}
\newcommand{\fett}[1]{\textbf{\boldmath\color{red!60!black}#1}}
%%%%%%%%%%%%%%%%%%%%%%%%%%%%%%%%%%%%%%%%%%%%%%%%%%%%%%%%%%%%%%%%%%%%%%%%%%%%

\begin{document}
\textit{\footnotesize For an English version of this exercise, see [\href{https://jeffe.cs.illinois.edu/teaching/algorithms/book/Algorithms-JeffE.pdf}{Erickson}, page 210]}.

Eine Klasse von Suchalgorithmen auf Graphen, die Breitensuche und Tiefensuche verallgemeinern, wurde 1975 von Edsger Dijkstra, Leslie Lamport, Alain Martin, Carel Scholten und Elisabeth Steffens beschrieben. Die Autor:innen haben diese Algorithmen untersucht, um damit einen automatischen \emph{garbage collector} zu entwerfen (siehe \href{https://de.wikipedia.org/wiki/Garbage_Collection}{wikipedia}).
Anstatt markierte und unmarkierte Knoten zu verwalten, verwaltet ihr Algorithmus eine Farbe für jeden Knoten, entweder weiß, grau oder schwarz.
Im Folgenden stellen wir uns einen als Adjazenzliste gegebenen, ungerichteten Graphen $G$ vor.

\begin{tabular}{p{0.4\textwidth}p{0.49\textwidth}}
\mbox{}\begin{algorithmic}
    \STATE{\textsc{ThreeColorSearch}$(s)$:}
    \STATE{färbe alle Knoten weiß}
    \STATE{färbe $s$ grau}
    \WHILE{mindestens ein Knoten ist grau}
        \STATE{\textsc{ThreeColorStep()}}
    \ENDWHILE{}
\end{algorithmic}
&
\mbox{}\begin{algorithmic}
    \STATE{\textsc{ThreeColorStep}$()$:}
    \STATE{$v\gets$ irgendein grauer Knoten}
    \IF{$v$ hat keine weißen Nachbarn}
        \STATE{färbe $v$ schwarz}
    \ELSE
        \STATE{$w\gets$irgendein weißer Nachbar von $v$}
        \STATE{$w.\pi\gets v$} \COMMENT{Hier speichern wir $v$ als Elternknoten von $w$.}
        \STATE{färbe $w$ grau}
    \ENDIF{}
\end{algorithmic}
\end{tabular}

\begin{enumerate}
    \item\label{Farbinvariante} Beweise, dass \textsc{ThreeColorSearch} zu jedem Zeitpunkt die folgende Invariante erhält: Kein schwarzer Knoten ist zu einem weißen Knoten benachbart. \emph{(Hinweis: Das sollte einfach sein.)}
    \item Beweise Folgendes: Wenn \textsc{ThreeColorSearch}$(s)$ terminiert, dann sind alle von $s$ erreichbaren Knoten schwarz, alle nicht von $s$ erreichbaren Knoten weiß, und die zu den Eltern zeigenden Kanten $(v, v.\pi)$ definieren einen Baum, der alle Knoten der Zusammenhangskomponente von $s$ aufspannt.
    \emph{(Hinweis: Wenn man den Algorithmus mit DFS/BFS vergleicht, kann man sich intuitiv vorstellen, dass die schwarzen Knoten \enquote{markiert} sind und die grauen Knoten \enquote{auf dem Stapel/in der Warteschlange}. Ein Unterschied ist, dass \textsc{ThreeColorStep} nicht im selben Aufruf \emph{alle} Kanten abarbeiten muss, die aus einem Knoten $v$ rausgehen.)}
    \item Die folgende Variante von \textsc{ThreeColorSearch} verwaltet die grauen Knoten auf einem Stapel.
    Beweise, dass diese Variante äquivalent zu DFS ist, das heißt, die Knoten werden in genau derselben Reihenfolge entdeckt und die Elternbeziehungen sind identisch.
    \emph{Hinweis: Die Reihenfolge der letzten zwei Zeilen von \textsc{ThreeColorStackStep} ist wichtig!}

    \begin{tabular}{p{0.4\textwidth}p{0.49\textwidth}}
        \mbox{}\begin{algorithmic}
            \STATE{\textsc{ThreeColorStackSearch}$(s)$:}
            \STATE{färbe alle Knoten weiß}
            \STATE{färbe $s$ grau}
            \STATE{\fett{lege $s$ auf den Stapel}}
            \WHILE{mindestens ein Knoten ist grau}
                \STATE{\textsc{ThreeColorStackStep()}}
            \ENDWHILE{}
        \end{algorithmic}
        &
        \mbox{}\begin{algorithmic}
            \STATE{\textsc{ThreeColorStackStep}$()$:}
            \STATE{\fett{nimm $v$ vom Stapel}}
            \IF{$v$ hat keine weißen Nachbarn}
                \STATE{färbe $v$ schwarz}
            \ELSE
                \STATE{$w\gets$irgendein weißer Nachbar von $v$}
                \STATE{$w.\pi\gets v$}
                \STATE{färbe $w$ grau}
                \STATE{\fett{lege $v$ auf den Stapel}}
                \STATE{\fett{lege $w$ auf den Stapel}}
            \ENDIF{}
        \end{algorithmic}
        \end{tabular}

    \item\label{notBFS} Die folgende Variante von \textsc{ThreeColorSearch} verwaltet die grauen Knoten in einer Warteschlange.
    Beweise, dass diese Variante \emph{\textbf{nicht}} äquivalent zu BFS ist.
    \emph{Hinweis: Die Reihenfolge der letzten zwei Zeilen von \textsc{ThreeColorQueueStep} ist nicht wichtig!}
    
        \begin{tabular}{p{0.4\textwidth}p{0.49\textwidth}}
            \mbox{}\begin{algorithmic}
                \STATE{\textsc{ThreeColorQueueSearch}$(s)$:}
                \STATE{färbe alle Knoten weiß}
                \STATE{färbe $s$ grau}
                \STATE{\fett{schiebe $s$ in die Warteschlange}}
                \WHILE{mindestens ein Knoten ist grau}
                    \STATE{\textsc{ThreeColorQueueStep()}}
                \ENDWHILE{}
            \end{algorithmic}
            &
            \mbox{}\begin{algorithmic}
                \STATE{\textsc{ThreeColorQueueStep}$()$:}
                \STATE{\fett{ziehe $v$ aus der Warteschlange}}
                \IF{$v$ hat keine weißen Nachbarn}
                    \STATE{färbe $v$ schwarz}
                \ELSE
                    \STATE{$w\gets$irgendein weißer Nachbar von $v$}
                    \STATE{$w.\pi\gets v$}
                    \STATE{färbe $w$ grau}
                    \STATE{\fett{schiebe $v$ in die Warteschlange}}
                    \STATE{\fett{schiebe $w$ in die Warteschlange}}
                \ENDIF{}
            \end{algorithmic}
            \end{tabular}

    \item (\veryhard) Hier sei $G$ ein gerichteter Graph. Wir nehmen nun an, dass ein zweiter Prozess Kanten zu $G$ hinzufügt, während \textsc{ThreeColorSearch} noch läuft. Diese neuen Kanten könnten die Farbinvariante zerstören, die in Teil~\ref{Farbinvariante} beschrieben ist. Daher könnte es jetzt sein, dass \textsc{ThreeColorSearch} nicht mehr korrekt ist. Das heißt, es könnte sein, dass der Algorithmus zwar terminiert, aber es trotzdem noch Knoten gibt, die von $s$ erreichbar sind und weiß sind. Wenn wir einen \emph{garbage collector} implementieren wollen, wäre das fatal, denn hier würden wir \enquote{weiß} mit \enquote{unerreichbar und daher löschbar} gleichsetzen wollen.
    
    Wenn der andere Prozess auf die Farbinvariante Rücksicht nimmt und diese explizit wiederherstellt, wann immer sie verletzt würde, dann können wir den \textsc{ThreeColorSearch} Algorithmus trotzdem sicher verwenden. Das möchten wir jetzt zeigen. Um die zwei parallelen Algorithmen zu modellieren, verwenden wir die \emph{either/or} Syntax in \textsc{GarbageCollect}; wie bei \emph{if/then/else} verzweigt das Programm hier, aber welcher Zweig verfolgt wird, wird nicht vom Programm selbst entschieden, sondern vom Betriebssystem.\footnote{Das ist eine dramatische Vereinfachung, sowohl von paralleler Programmierung als auch von \emph{garbage collection}. Mehrfädige Programmiersprachen wie Lua und Go benutzen einen viel komplexeren \emph{mark and sweep} Algorithmus als \emph{garbage collector}. Mathematisch wichtig für \emph{either/or} ist, dass das Betriebssystem zwar nicht garantiert, wie oft hintereinander und in welcher Reihenfolge die zwei Programmzweige gewählt werden. Aber jeder Zweig wird immer wieder \emph{irgendwann} gewählt, das heißt, nach endlicher Zeit. --- Das Betriebsystem darf also \emph{\textbf{nicht}} für immer den einen Zweig wählen und den anderen \enquote{vergessen}.}

    \begin{tabular}{p{0.4\textwidth}p{0.49\textwidth}}
        \mbox{}\begin{algorithmic}
            \STATE{\textsc{GarbageCollect}$(s)$:}
            \STATE{färbe alle Knoten weiß}
            \STATE{färbe $s$ grau}
            \WHILE{mindestens ein Knoten ist grau}
                \STATE{\fett{either}}
                    \STATE{\quad\textsc{CollectStep()}}
                \STATE{\fett{or}}
                    \STATE{\quad\fett{\textsc{Mutate()}}}
            \ENDWHILE{}
        \end{algorithmic}
        &
        \mbox{}\begin{algorithmic}
            \STATE{\textsc{CollectStep}$()$:}
            \STATE{$v\gets$ irgendein grauer Knoten}
            \IF{$v$ hat keine weißen Nachbarn}
                \STATE{färbe $v$ schwarz}
            \ELSE
                \STATE{$w\gets$irgendein weißer Nachbar von $v$}
                \STATE{färbe $w$ grau}
            \ENDIF{}
        \end{algorithmic}\\
        \mbox{}\begin{algorithmic}
            \STATE{\textsc{Mutate}$()$:}
            \STATE{$u\gets$ irgendein Knoten}
            \STATE{$w\gets$ irgendein Knoten}
            \IF{$(u,w)$ ist keine Kante}
                \STATE{füge $(u,w)$ als Kante hinzu}
                \IF{$u$ ist schwarz und $w$ ist weiß}
                \STATE{färbe $u$ grau}
                \ENDIF
                \IF{$u$ ist weiß und $w$ ist schwarz}
                \STATE{färbe $w$ grau}
                \ENDIF
            \ENDIF{}
        \end{algorithmic}
        \end{tabular}

        Beweise, dass \textsc{GarbageCollect} irgendwann terminiert, und dass dann jeder von $s$ erreichbare Knoten schwarz gefärbt ist und jeder von $s$ nicht erreichbare Knoten weiß gefärbt ist.
    \item (\veryhard) Hier sei $G$ wieder ein gerichteter Graph. Anstatt schwarze Knoten grau zu färben, soll \textsc{Mutate} jetzt die Farbinvariante aufrechterhalten, indem manche \emph{weißen} Knoten grau gefärbt werden:
    \begin{algorithmic}
        \STATE{\textsc{Mutate}$()$:}
        \STATE{$u\gets$ irgendein Knoten}
        \STATE{$w\gets$ irgendein Knoten}
        \IF{$(u,w)$ ist keine Kante}
            \STATE{füge $(u,w)$ als Kante hinzu}
            \IF{$u$ ist schwarz und $w$ ist weiß}
            \STATE{\fett{färbe $w$ grau}}
            \ENDIF
            \IF{$u$ ist weiß und $w$ ist schwarz}
            \STATE{\fett{färbe $u$ grau}}
            \ENDIF
        \ENDIF{}
    \end{algorithmic}
    Beweise, dass \textsc{GarbageCollect} irgendwann terminiert, und dass dann $s$ schwarz gefärbt ist, jeder von einem schwarzen Knoten erreichbare Knoten wiederum schwarz ist, und jeder nicht von einem schwarzen Knoten erreichbare Knoten weiß ist.
\end{enumerate}

\textbf{Hinweise zur Abgabe.}
Die Abgabe soll wie immer per PDF erfolgen, allerdings sind diesmal ausschließlich mathematische Beweise gefordert. Um einen \emoji{star} zu erhalten, müssen~\ref{Farbinvariante} bis~\ref{notBFS} zielgerichtet, nachvollziehbar, lesbar, vollständig, und korrekt gelöst sein.
Zielgerichtet heißt, dass alle Sätze und Satzteile in der Abgabe aktiv zum Ziel hinführen, nämlich der Lösung der jeweiligen Aufgabe.
\end{document}
